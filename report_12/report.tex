%%%%%%%%%%%%%%%%%%%%%%%%%%%%%%%%%%%%%%%%%
% University/School Laboratory Report
% LaTeX Template
% Version 3.0 (4/2/13)
%
% This template has been downloaded from:
% http://www.LaTeXTemplates.com
%
% Original author:
% Linux and Unix Users Group at Virginia Tech Wiki 
% (https://vtluug.org/wiki/Example_LaTeX_chem_lab_report)
%
% License:
% CC BY-NC-SA 3.0 (http://creativecommons.org/licenses/by-nc-sa/3.0/)
%
%%%%%%%%%%%%%%%%%%%%%%%%%%%%%%%%%%%%%%%%%

%----------------------------------------------------------------------------------------
%	PACKAGES AND DOCUMENT CONFIGURATIONS
%----------------------------------------------------------------------------------------

\documentclass{article}
\usepackage{geometry}
 \geometry{
 a4paper,
 total={210mm,297mm},
 left=20mm,
 right=20mm,
 top=20mm,
 bottom=20mm,
 }

\usepackage{siunitx} % Provides the \SI{}{} command for typesetting SI units
\usepackage{listings}
\usepackage{graphicx} % Required for the inclusion of images
\usepackage{enumerate}
\usepackage{float}
\usepackage{fancyvrb}
\usepackage[utf8]{inputenc}

\setlength\parindent{0pt} % Removes all indentation from paragraphs

%\usepackage{times} % Uncomment to use the Times New Roman font

%----------------------------------------------------------------------------------------
%	DOCUMENT INFORMATION
%----------------------------------------------------------------------------------------

\title{Network Security \\ 389.159 - SS 2018 \\ Lab Exercise 1 \& Lab Exercise 2} % Title

\author{
    TEAM 02 \\
    Corentin \textsc{Bergès} (11741629) (xxxx) \\
    Christoph \textsc{Echtinger-Sieghart} (00304130) (066 938)
}

\date{\today} % Date for the report
\begin{document}

\maketitle % Insert the title, author and date

%----------------------------------------------------------------------------------------
%	SECTION 1
%----------------------------------------------------------------------------------------
\renewcommand{\arraystretch}{2} %Stretch rows
\section{Lab Exercise 1}

\subsection{rep-1}

We obtained our IPv4 address using the following command:

\begin{Verbatim}
team02@pc01:~$ ip address show dev em1
\end{Verbatim}

Our IPv4 address is \textbf{192.168.83.20}.

\subsection{rep-2}

We sent TCP SYN packets to IPv4 address \textbf{192.168.83.33} on ports 113 (ident), 445 (microsoft-ds),
7210 and 9920 using the following command:

\begin{verbatim}
team02@pc01:~$ for port in 113 445 7210 9920; do \
        tgn "ip(dst = 192.168.83.33) /tcp(dst = $port,syn)"; \
    done
\end{verbatim}

We now present our findings for each port. We performed the analysis using \texttt{wireshark}, but for
brevity include only a textual representation of the communication. Note that the textual representation was
obtained using \texttt{tcpdump} and has been truncated to show only important data.

\paragraph{Port 113}
\begin{Verbatim}
IP 192.168.83.20.1073 > 192.168.83.33.113: Flags [S], seq 0, win 8192, length 0
IP 192.168.83.33.113 > 192.168.83.20.1073: Flags [R.], seq 0, ack 1, win 0, length 0
\end{Verbatim}

This communication corresponds to \textbf{case B}. Port 113 is closed on the target machine.

\paragraph{Port 445}

\begin{Verbatim}
IP 192.168.83.20.1073 > 192.168.83.33.445: Flags [S], seq 0, win 8192, length 0
IP 192.168.83.33.445 > 192.168.83.20.1073: Flags [S.], seq 4122740187, ack 1, win 29200
IP 192.168.83.20.1073 > 192.168.83.33.445: Flags [R], seq 1, win 0, length 0
\end{Verbatim}

This communication corresponds to \textbf{case A}. Port 445 is open on the target machine. This
is the case with the different behaviour. The difference is due to our TCP/IP stack responding with
a RST packet because there is no longer a connection associated with the initial packet.

\paragraph{Port 7210}
\begin{Verbatim}
IP 192.168.83.20.1073 > 192.168.83.33.7210: Flags [S], seq 0, win 8192, length 0
\end{Verbatim}

This communication corresponds to \textbf{case C}. The packet to port 7210 was silently dropped.

\paragraph{Port 9920}
\begin{Verbatim}
IP 192.168.83.20.1073 > 192.168.83.33.9920: Flags [S], seq 0, win 8192, length 0
IP 192.168.83.33 > 192.168.83.20: ICMP 192.168.83.33 tcp port 9920 unreachable, length 48
\end{Verbatim}

This communication corresponds to \textbf{case D}. We got an ICMP message back, telling us that port 9920
is blocked.

\subsection{rep-3}

Depending on the amount of traffic passing through the routing device it would be hard to
detect any kind of scanning activity on a busy router. Wireshark is not a suitable tool for analyzing
large amounts of network traffic data, because \ldots
\begin{itemize}
    \item the traffic dump would quickly get quite large, because the packet payload is captured as well.
    \item Wireshark does not provide suitable binning or aggregation functions.
    \item we often need statistical/data mining methods to extract useful information from the traffic dump.
\end{itemize}

\section{Lab Exercise 2}

\subsection{rep-4}

Figure~\ref{figure:command_3_rows} shows the commandline and the first three rows from
our \textbf{synscan.pcap} file. Note that the \texttt{tcpdump} output is truncated on the
right side.

\begin{figure}[h]
\begin{Verbatim}[fontsize=\small]
team02@pc01:~$ cat synscan.pcap | tcpdump -t -nr - | head -n 3
IP 192.168.83.20.850 > 192.168.83.1.2049: Flags [P.], seq 4210818527:4210818687, ack 463369472, win 4229, options [nop,nop,TS val 148271 ecr 1080118146], length 160: NFS request xid 3254235248 156 getattr fh 0,0/22
IP 192.168.83.1.2049 > 192.168.83.20.850: Flags [P.], seq 1:129, ack 160, win 6365, options [nop,nop,TS val 1080118157 ecr 148271], length 128: NFS reply xid 3254235248 reply ok 124 getattr NON 3 ids 0/3 sz 0
IP 192.168.83.20.850 > 192.168.83.1.2049: Flags [.], ack 129, win 4229, options [nop,nop,TS val 148271 ecr 1080118157], length 0
\end{Verbatim}
\caption{\label{figure:command_3_rows} First three rows of our synscan.pcap file}
\end{figure}

\subsection{rep-5}

We converted our synscan.pcap file to a flow record using the following command:

\begin{Verbatim}
team02@pc01:~$ rwptoflow --flow-output=synscan.rw synscan.pcap --compression-method=zlib
\end{Verbatim}

Figure~\ref{figure:command_5_rows} shows the commandline and the first five rows from our
\textbf{synscan.rw} flow record file.

\begin{figure}[h]
\begin{Verbatim}
team02@pc01:~$ rwcut --fields sTime,sIp,dIP,sPort,dPort,ttl,flags synscan.rw | head -n 6
                  sTime|                 sIP|                 dIP|sPort|dPort|ttl|   flags|
2018/05/15T11:09:44.616|       192.168.83.20|        192.168.83.1|  850| 2049| 64|   PA   |
2018/05/15T11:09:44.616|        192.168.83.1|       192.168.83.20| 2049|  850| 64|   PA   |
2018/05/15T11:09:44.616|       192.168.83.20|        192.168.83.1|  850| 2049| 64|    A   |
2018/05/15T11:09:44.616|       192.168.83.20|        192.168.83.1|  850| 2049| 64|   PA   |
2018/05/15T11:09:44.616|        192.168.83.1|       192.168.83.20| 2049|  850| 64|   PA   |
\end{Verbatim}
\caption{\label{figure:command_5_rows} First five rows of our synscan.rw file}
\end{figure}

\subsection{rep-6}

The start time of our team02.flowrecord.rw file is \textbf{2012/04/26T20:00:00,38257}. This corresponds to
a unix epoch of \textbf{1335470400,38257}. The end time is \textbf{2012/04/26T20:59:59,34414}. This corresponds
to a unix epoch of \textbf{1335473999,34414}.

\subsection{rep-7}

Figure~\ref{figure:packets_per_hour} shows the commandline and the number of packets per hour in our
team02.flowrecord.rw file.
The are 135358046 packets per hour in our file. This number is in the
same order of magnitude as the number in the exercise sheet ($\approx 10^8 \text{ to } 10^9$ packets/hour).

\begin{figure}[h]
\begin{verbatim}
team02@pc01:~$ rwuniq --sort-output --bin-time=3600 --fields=stime \
    --values=packet --timestamp-format=epoch --no-titles \
    --no-final-delimiter --delimited=, workfiles/team02.flowrecord.rw
1335470400,135358046
\end{verbatim}
\caption{\label{figure:packets_per_hour} Number of packets per hour in team02.flowrecord.rw}
\end{figure}

\subsection{rep-8}

Figure~\ref{figure:unique_ip_per_hour} shows the commandline and the number of unique IP addresses per hour in
our team02.flowrecord.rw file.
The are 447148 unique IP addresses per hour in our file. This number is in the
same order of magnitude as the number in the exercise sheet ($\approx 10^5 \text{ to } 10^6$ unique IPs/hour).

\begin{figure}[h]
\begin{verbatim}
team02@pc01:~$ rwuniq --sort-output --bin-time=3600 --fields=stime \
    --values=sIP --timestamp-format=epoch --no-titles \
    --no-final-delimiter --delimited=, workfiles/team02.flowrecord.rw
1335470400,447148
\end{verbatim}
\caption{\label{figure:unique_ip_per_hour} Unique IP addresses per hour in team02.flowrecord.rw}
\end{figure}

\subsection{rep-9}

Since the darkspace is a /8 network it is approximately $\frac{1}{256}^{\text{th}}$ of the whole internet.
To be more accurate all publicly unusable address ranges would have to be taken into account.

\paragraph{Undesired packets/hour}
Using this information we arrive at
\[
    135358046 * 256 = 34651659776 \approx 34e9
\] undesired packets per hour for the whole internet. The real values are probably smaller, because
the publicly unusable address ranges have not been taken into account.

\paragraph{Unique source IPs}
The number of unique source IPs will roughly be the same for the whole internet as for the darkspace
($\approx 4e5$ unique IPs per hour).  The real values are probably bigger, because attackers might
avoid the darkspace or target only specific ranges of the address space.

\paragraph{Pollution}

To give a reliable answer, we would need actual data on the average packet count per hour for the
whole internet. We checked caida.org, but could not find suitable data. The traffic statistics at DE-CIX or
AMS-IX only give bandwith and not packet count. Since an IP packet might range from 64 bytes to 1500 bytes transported over ethernet deriving a packet count from bandwidth stats is not reliable.

One major effect of the unwanted packets is that the unwanted traffic uses up
valuable bandwidth and resources. The unwanted traffic also results in increased power consumption.

\end{document}
